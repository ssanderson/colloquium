\documentclass[twoside]{report}
\usepackage[top=1.0in, bottom=1in, left=1.5in, right=1in, includehead]{geometry}
\usepackage{thesiscommands}
\usepackage{appendix}

\pagestyle{headings}

%% Include some fancy-schmancy macros
\include{thesis-macros}

%%%%%%%%%%%%%%%%%%%%%%%%%%%%%
%% Thesis body             %%
%%%%%%%%%%%%%%%%%%%%%%%%%%%%%

\begin{document}

%%%%%%%%%%%%%%%%%%%%%%%%%%%%%
%% Title page              %%
%%%%%%%%%%%%%%%%%%%%%%%%%%%%%
\begin{titlepage}
$\;$
\vskip1.5in
\begin{center}
{\LARGE
  {\bf Hilbert's Nullstellensatz: An Introduction to Algebraic Geometry}\\
}
\large
\vskip.25in
by Scott B. Sanderson\\
\vskip.125in
Susan Loepp, Advisor\\
\vskip.5in
\small

\vskip.5in
Williams College\\
Williamstown, Massachusetts\\
\vskip.5in
\today
\vskip.5in
{\Huge \textbf{DRAFT}}
\end{center}
\end{titlepage}
%%%%%%%%%%%%%%%%%%%%%%%%%%%%%

% \tableofcontents
% \listoffigures
% \listoftables
% \onehalfspacing

\newcommand{\poly}[1]{#1[x]}
\newcommand{\polyn}[1]{#1[x_1, x_2,\cdots,x_n]}
\newcommand{\ideal}[1]{\langle#1\rangle}
\newcommand{\varideal}[1]{\mathcal{I}(#1)}
\newcommand{\variety}[1]{\mathbf{V}(#1)}

\addtocounter{chapter}{1}

\section{Ring Theory Basics}

\setlength{\parindent}{0pt}

\subsection{Definition of a Ring}

A ring $R$ is a set of elements along with a pair of binary
operations, $(+, \times)$, which we generally call addition and
multiplication.  By convention, we often write $a \times b$ as $ab$.
These operations must satisfy the following conditions:

\begin{itemize}
\item \textbf{Closure}: $a+b \in R$ and $a \times b \in R$.
\item \textbf{Associativity}: $a+(b+c)=(a+b)+c$ and $a(bc) = (ab)c$.
\item \textbf{Commutativity under Addition}: $a+b = b+a$
\item \textbf{Additive Identity}: There is a unique element, $0$, such that $a+0=a$.
\item \textbf{Additive Inverses}: For every $a \in R$, there exists a
  unique element,$-a$, such that $a + (-a) = 0$. (By convention, we
  often write the expression $a + (-b)$ as simply $a-b$.
\item \textbf{Distributativity}: $a(b+c) = ab + ac$ and $(a+b)c = ac
  + bc$.
\end{itemize}

\bigskip For this talk, the rings that we care about will have two
more important properties:

\begin{itemize}
\item \textbf{Commutativity under Multiplication}: $ab = ba$.
\item \textbf{Multiplicative Identity}: There exists a unique
  element, $1$, such that $1a = a1 = a$.
\end{itemize}

Rings satisfying the above properties are called \textbf{commutative
  rings} with \textbf{unity}.\\

Some of the rings we consider will have one further property:

\begin{itemize}
\item \textbf{Multiplicative Inverses}: For every $a \in R, a \neq 0$, there
  exists a unique element, $a'$ such that $aa' = a'a = 1$
\end{itemize}

Rings satisfying all of the above properties are known as
\textbf{fields}.\\

Examples of fields include: $\rationals$, $\reals$, $\complexes$ with
the usual $+$ and $\times$ and $\integers_p$, the integers with
addition and multiplication modulo some prime, $p$. Examples of
commutative rings with unity that are not fields include $\integers$,
and $\poly{\reals}$, the ring of polynomials in one variable with
real-valued coefficients.

\subsection{Ideals}

A subset, $I$ of ring $R$ is an ideal of $R$ if the following conditions hold:

\begin{itemize}
\item For any $a, b \in I$, $a + b \in I$.
\item For any $a \in I$ and $r \in R$, $ar \in I$.
\end{itemize}

Thus ideals are subsets of $R$ which are closed under addition with
their own elements as well as under multiplication with \textit{any}
element of $R$.\\

Given a set of elements $S \subset R$, we define the ideal generated
by $S$, $\ideal{S}$ to be the smallest ideal containing $S$.  It is
not too hard to verify that 
$$\ideal{S} = \set{r \mid r = r_1s_1 + r_2s_2 + \cdots + r_ks_k}$$ 
where $s_i \in S$ and $r_i \in R$.\\

\textbf{Example:} Over the ring of integers, the ideal generated by
any element $n$ is $n\integers$, the subring containing all integers
divisible by $n$.\\

Given any ideal $I \subseteq R$ we can construct the \textbf{radical
  ideal} $\sqrt{I}$ of $I$, given by\\

\centerline{$\sqrt{I} = \set{r \mid r^n \in I}$, $r \in R, n \in
  \naturals$} \vspace{\baselineskip}

\textbf{Example:} Over the ring of integers, the subset $n\integers$
(defined above) is an ideal, and $\sqrt{n\integers}$ = $k\integers$,
where $k$ is the product of all the prime factors of
$n$.\\

\textbf{Note:} An important consequence of the closure rules above is
that any ideal $I \subseteq R$ which contains the identity element of
$R$ contains every element of $R$, since any $r \in R$ can be written
in the form $1r$.

\subsection{Polynomial Rings}

Given a ring $R$, we can define an associated ring, $\poly{R}$, called
the \textbf{polynomial ring} over $R$ in the indeterminate, $x$.
Elements of such a ring are of the form: $r_nx^n + r_{n-1}x^{n-1} +
\cdots + r_1x^1 + r_0$, where each $r_i$ is an element of $R$.  The
$x_i$ are treated simply as formal symbols. Addition and
multiplication are defined to correspond with the rules for addition
and multiplication of ``traditional'' polynomials. \\

Given any $r \in R$, we can define a natural homomorphism, $\phi_r:
\poly{R} \rightarrow R$ which corresponds to ``evaluating'' each
polynomial in $R[x]$ with the argument $r$.  More specifically, if
$p(x) \in \poly{R} = c_nx^n + c_{n-1}x^{n-1} + \cdots + c_1x^1 + c_0$,
then:
$$\phi_r(p) = c_nr^n + c_{n-1}r^{n-1} + \cdots c_1r + c_0$$  
(Here the notation $r^n$ means $r$ multiplied by itself $n$ times.)\\

These notions generalize to polynomials defined over any number of
indeterminates, given by $\polyn{R}$.  Addition and multiplication are
defined in the usual way, and for each $x \in R^n$ we can define an
associated homomorphism $\phi_{x}$ that corresponds to plugging in
values for each coordinate of $x$.\\

\textbf{Note:} When no confusion will arise we often denote
$\phi_x(p)$ by the simpler notation $p(x)$.  It is important to
recognize that this is not an alternative notation for the polynomial
$p$ in an indeterminate $x$, but rather stands for the value given by
evaluating $p$ at the point $x$.  In particular, we note that $p$ will
denote an element of $\polyn{R}$, $x$ will denote an element of $R^n$,
and $p(x)$ will denote an element of $R$.

Often we are particularly interested in the values for which a
polynomial $p$ evaluates to zero (that is to say, the values $r \in
R^n$ for which $\phi_r(p) = 0$).  We refer to the set of these values
as the \textbf{zero set} of $p$.  A field $F$ is said to be
\textbf{algebraically closed} if, for every $p \in \polyn{F}$, we have
that the zero set of $p$ is nonempty.\\

If we have some set $S = \set{p_1, p_2,\cdots, p_k} \subseteq
\polyn{F}$, we say that the intersection of the zero sets of the $p_i$
is the \textbf{affine variety} associated with $S$, and we denote this
set by $\variety{S}$.

\section{Connecting Varieties and Ideals}

\begin{lemma} If $S \subseteq S' \subseteq \polyn{R}$, then
  $\variety{S'} \subseteq \variety{S}$.
  
  Let $x \in \variety{S'}$, and let $p \in S$.  By stipulation, $p \in
  S'$, which implies that $p(x) = 0$, so $x \in \variety{S}$.
\end{lemma}

We find that it is sometimes difficult to remember the direction of
inclusion in the above.  A useful mnemonic is ``big ideals have small
varieties''.

\begin{proposition} For any set $S \subset \polyn{K}$, $\variety{S} = \variety{\ideal{S}}$\\

  Since $S \subseteq \ideal{S}$, we have $\variety{\ideal{S}}
  \subseteq \variety{S}$, by the previous proposition.\\

  Let $x \in \variety{S}$. We want to show that $x \in
  \variety{\ideal{S}}$, i.e., that every polynomial in $\ideal{S}$
  vanishes on $x$.  Let $p \in \ideal{S}$.  Then $p$ can be written
  $p_1q_1 + p_2q_2 + \cdots + p_nq_n$, where each $p_i \in S$, and
  each $q_i \in \polyn{K}$.  It follows that $p(x) = p_1(x)q_1(x) +
  p_2(x)q_2(x) + \cdots + p_n(x)q_n(x)$, and since each $p_i \in S$,
  each $p_i(x) = 0$, which implies that $p(x) = 0$, as desired.

\end{proposition}

\begin{proposition} For any set $V \subset K$, the set of polynomials
  $\varideal{V} \subseteq \polyn{K}$ that vanish on $V$ form an ideal
  whose variety is $V$.\\

  It suffices to show that $\varideal{V}$ is closed under addition and
  closed under multiplication by any element of $K$.  This shows that
  $\varideal{V}$ is a subring of $K$, since all the other requirements
  are trivially inherited by elements of $\varideal{V}$ as elements of
  $\polyn{K}$, so by definition it also shows that $\varideal{V}$ is
  an ideal.\\

  Let $p$ and $q$ be polynomials in $\varideal{V}$, and let $x$ be an
  element of $V$. Then $\phi_x(p) = \phi_x(q) = 0$, so $\phi_x(p + q)
  = \phi_x(p) + \phi_x(q) = 0$.  Similarly, let $p \in \varideal{V}$,
  and let $q \in \polyn{K}$.  Then $\phi_x(p)\phi_x(q) = 0\phi_x(q) =
  0$.
\end{proposition}

The latter proposition justifies our language in saying that
$\varideal{V}$ is the ideal associated with the variety $V$.  Using
the definitions above we can also note that in general $I \subseteq
\varideal{\variety{I}}$ (There may be an ideal larger than $I$ that
still vanishes on every point in the zero set of $I$.), and $S
\subseteq \variety{\varideal{S}}$ (Again, it may be the case that
every polynomial in $\polyn{K}$ that vanishes everywhere in $S$ also
vanishes on some other point $x \notin S$.)

\section{Hilbert's Nullstellensatz: Weak and Strong}

\begin{theorem} \textbf{Hilbert Nullstellensatz (Weak Form)}\\ 

  Let $K$ be an algebraically closed field, and let $I \subseteq
  \polyn{K}$ be an ideal such that $\variety{I} = \emptyset$.  Then $I =
  \polyn{K}$.
\end{theorem}

\begin{theorem} \textbf{Hilbert Nullstellensatz (Strong Form)}\\ 

  Let $K$ be an algebraically closed field, and let $I \subseteq
  \polyn{k}$.  Then $\varideal{\variety{I}} = \sqrt{I}$.
\end{theorem}

We first show that the strong form implies the weak.  Let $K$ be an
algebraically closed field and $I$ an ideal in $\polyn{K}$ with
$\variety{I} = \emptyset$, and suppose that $\varideal{\variety{I}} =
\sqrt{I}$.  Since $\variety{I} = \emptyset$, $\varideal{\variety{I}}$
is the set of polynomials that vanish on every point in the empty set.
But this statement is (vacuously) true for \emph{all} polynomials in
$\polyn{K}$, so $\varideal{\variety{I}} = \polyn{K}$.  So we have that
$\sqrt{I} = \polyn{K}$, which implies that $1 \in \sqrt{I}$, so there
exists an $n \in \naturals$ such that $1^n \in I$, so $1 \in I$, which
implies that $I = \polyn{K}$.


We now show that the weak version of the Nullstellensatz is actually
sufficient to prove the strong version.




  


  
%%%%%%%% References %%%%%%%%%%
\bibliographystyle{acm}
\bibliography{references}

%%%%%%%% End References %%%%%%

\end{document}

